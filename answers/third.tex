{\bf Answer:}

{\bf Part a:} \citep{kaplan2020scaling} is a study on neural scaling laws that investigated the effect of variables such as the model architecture, the model size (the network's width and depth for, example), the computing power used to train, and the amount of training dataset on the overall performance. While related work exists that explored such trends in models such as random forests~\citep{biau2012analysis} and image models~\citep{tan2019efficientnet}, this study focused on language models. The authors in~\citep{kaplan2020scaling} investigated and reported power-law relationships between the performance (in terms of the loss on the test set) of the classical Transformer model~\citep{vaswani2017attention} on the WebText2 dataset~\citep{radford2019language} and three main variables of interest; the number of model parameters $N$, the dataset size $D$, and the amount of compute required $C$. Across the variety of experiments conducted, the following trends are reported in the paper:
\begin{enumerate}
\item {\bf Power laws:} When not bottlenecked by the other two, the accuracy has a power-law relationship with each of the three factors $N, D, C$.
\item {\bf Overfitting:} If suboptimal values of $N$ or $D$ are used, the accuracy incurs a penalty as the scaling variable is increased.
\item {\bf Model size:} Large models tend to perform better overall; they need fewer optimization steps and a smaller dataset to achieve the same accuracy as smaller models. In fact, according to the authors the optimal performance is reached by training very large models and stopping significantly short of the convergence criteria.
\item {\bf Transfer learning:} Interestingly, the model's performance on a dataset from a different distribution than the training one is strongly correlated to the performance on the training set but with a constant offset.
\end{enumerate}
%
Since these scaling laws do not assume any special properties about the functions they are approximating, it is reasonable to expect these trends to carry over to SciML applications. While it is hard to find such studies done for physics-informed neural networks (architectures that can learn a given PDE solution function), a number of scaling studies are done in the field of operator learning where many operators of interest are solution operators of PDEs. For example, the original DeepONet (deep operator network) paper~\citep{lu2019deeponet} showed various error convergence rates as functions of the the network width, the amount of training data (number of functions), and the number of ``sensor locations'' (locations where the input functions are sampled).~\citep{de2022cost} reported a power-law relationship between the in- and out-of-distribution accuracy and the size of the training dataset for DeepONets, FNOs, PCA-Nets, and PARA-Nets.~\citep{lanthaler2022error} studied the effect of the DeepONet network size on its approximation and generalization errors for specific operators. Finally,~\citep{liu2024neural} presented neural scaling laws (observed to be power laws) for DeepONets' approximation and generalization errors for general Lipschitz operators with respect to the model and dataset sizes. While according to the scaling laws for operator learning methods the model error can be driven down arbitrarily, the empirical results from~\citep{lu2019deeponet} showed that this is not true. For example, Figure 6 in the paper showed that the relationship between training and test error with respect to the number of sensor locations initially follows a power law but plateaus after the number exceeds $10^1$. We see similar trends with the network width in Figure 2. This is consistent with the insights put forth by~\citep{kaplan2020scaling}, that most scaling power laws plateau.

{\bf Part b:}~\citep{garg2022can} studied in-context learning of function classes with transformers. In-context learning is the ability of a model to, for a {\it previously unseen} function $f$, accurately predict $f(x_{i+1})$ at query point $x_{i+1}$ when given a prompt of $i$ ``in context'' sampling locations and samples of $f$; $\left(x_1, f(x_1), x_2, f(x_2), \dots, x_i, f(x_i) \right)$. For the class of linear functions, the transformer model can in-context learn well enough to be on par with several baselines; least squares estimator (known to be optimal), n-nearest neighbors, and averaging. In-context learning is also robust, it performs well even when the in-context outputs ($f(x_1), f(x_2), \dots, f(x_i)$) are noisy at inference time and when the distribution of the in-context examples and the query differ. The transformer model is also shown to in-context learn other function classes, mainly sparse linear functions, decision trees, and shallow neural networks. \citep{zhang2024trained} went a step further and showed how in-context learning is ``learning a learning algorithm from data'' and studied different distribution shifts under which the standard approach fails. Section 4.2 revisits three such shifts; task ($\D_{f}^{\text{train}} \neq \D_{f}^{\text{test}}$), query ($\D_{\text{query}}^{\text{test}} \neq \D_{x}^{\text{test}}$), and covariate ($\D_{x}^{\text{train}} \neq \D_{x}^{\text{test}}$). Transformers handle well the task and query shifts (the latter as long as $\D_{x}^{\text{train}} = \D_{x}^{\text{test}}$) but not the covariate shifts. Interestingly, the same covariate shift phenomenon has been observed in SciML!

\citep{lu2019deeponet} stated clearly that in order for DeepONets to work, the sensor locations must be constant across all instances, essentially requiring $\D_{x}^{\text{train}} = \D_{x}^{\text{test}}$. This is generally true for most operator learning methods, though there is some work exploring ways to mitigate this by learning function encoders~\citep{ingebrand2025basis, zhang2023belnet}. Most operator learning architectures also suffer from task shifts but none from query shifts (either at training or test time). When learning solution operators of PDEs, often the input functions are different initial and/or boundary conditions. As such, in order to avoid suffering from task shifts, the instances of input functions across training and test sets need to come from the same underlying distribution.
%
\pagebreak
