\documentclass[12pt]{exam}
%\documentclass[answers,addpoints,12pt]{exam}
%\usepackage[utf8]{inputenc}

%\usepackage[gray]{xcolor}
\usepackage{amsmath,amssymb}
\usepackage{minted}
\usepackage{soul}
\usepackage{natbib}
\usepackage{tikz}
\usepackage[shortlabels]{enumitem}
\usepackage{graphics}
\usepackage{graphicx}
\usepackage{epsfig}
\usepackage{color}
\usepackage{amsfonts}
\usepackage{amsmath}
\usepackage{amsthm}
\usepackage{url}
\usepackage{epstopdf}
\usepackage[top=1in, bottom=1.5in, left=1in, right=1in]{geometry}

\usepackage{booktabs}       % professional-quality tables
\usepackage{nicefrac}       % compact symbols for 1/2, etc.
\usepackage{mathtools}
\usepackage{caption}
\usepackage{array}
\usepackage{tabularx}
\usepackage{cleveref}
\usepackage{arydshln}

\def\h{\mathfrak{h}}
\def\ve{\varepsilon}
\def\low{\mathrm{low}}

\def\ve{\varepsilon}
\def\d{\mathfrak{d}}
\def\db{{\bf d}}
\def\P{\mathbb{P}}
\def\R{\mathbb{R}}
\def\Rc{\mathcal{R}}
\def\cT{\mathcal{T}}
\def\cN{\mathcal{N}}
\def\S{\mathcal{S}}
\def\cP{\mathcal{P}}
\def\K{\kappa}
\def\Y{{\bf\mathcal{Y}}}
\def\Z{\mathcal{Z}}
\def\E{\mathbb{E}}
\def\Tor{\mathbb{T}}
\def\De{\mathbb{De}}
\def\Inv{\operatorname{Inv}}
\def\re{\text{re}}
\def\im{\text{im}}

\def\M{\mathcal{M}}
\def\Me{\mathbb{\mathcal{M}}}
\def\Mes{\mathcal{M}}
\def\C{{\mathbb{C}}}
\def\Pr{\mathcal{P}}
\def\cC{\mathcal{C}}
\def\sk{{\mathcal{K}}}
\def\sy{{\mathcal{S}}}
\def\cR{\mathcal{R}}
\def\B{\mathcal{B}}
\def\Bf{\mathfrak{B}}

\def\bx{\mathbf{x}}
\def\by{\mathbf{y}}
\def\bc{\mathbf{c}}
\def\bd{\mathbf{d}}
\def\bg{\mathbf{g}}

\def\ED{\mathcal{E}}
\def\Rh{\mathcal{R}_h}
\def\Rht{\mathcal{R}_{h,t}}
\def\er{\mathcal{E}}

\def\f{\mathcal{f}}

\def\one{\mathrm{I}}
\def\two{\mathrm{II}}

\def\St{\mathcal{S}}
\def\X{{\bf\mathcal{X}}}
\def\F{\mathcal{F}}
\def\D{\mathcal{D}}
\def\U{\mathcal{U}}
\def\V{\mathcal{V}}
\def\G{\mathcal{G}}
\def\I{\mathcal{I}}
\def\J{\mathcal{J}}
\def\L{\mathcal{L}}
\def\A{\mathcal{A}}
\def\W{\mathfrak{W}}
\def\cW{\mathcal{W}}
\def\Q{\mathfrak{Q}}
\def\Qm{{\bf Q}}
\def\H{\mathcal{H}}
\def\T{\mathcal{T}}
\def\Td{\top}
\def\O{\mathcal{O}}
\def\N{\mathcal{N}}
\newcommand{\todo}{{\color[rgb]{1,0.1,0.1} {\bf TO DO:}}}
\def\restrict#1{\raise-.5ex\hbox{\ensuremath|}_{#1}}

\def\<{\big\langle}
\def\>{\big\rangle}
\def\Img{\operatorname{Im}}
\def\Ker{\operatorname{Ker}}
\def\Cond{\operatorname{Cond}}
\def\Osc{\operatorname{Osc}}
\def\Proj{\operatorname{Proj}}
\def\Vol{\operatorname{Vol}}
\def\diiv{\operatorname{div}}
\def\dist{\operatorname{dist}}
\def\Var{\operatorname{Var}}
\def\Tr{\operatorname{Tr}}
\def\Trun{\operatorname{Trun}}
\def\Cov{\operatorname{Cov}}
\def\Card{\operatorname{Card}}
\def\det{\operatorname{det}}
\def\Hess{\operatorname{Hess}}
\def\diam{\operatorname{diam}}
\def\sym{{\operatorname{sym}}}
\def\diag{{\operatorname{diag}}}
\def\argmin{{\operatorname{argmin}}}
\def\sob{{\textrm{sob}}}
\def\lsob{{\textrm{l}}}
\def\esssup{{\operatorname{esssup}}}
\def\supp{{\operatorname{support}}}
\def\support{{\operatorname{supp}}}
\def\dim{{\operatorname{dim}}}
\def\atan{{\operatorname{atan}}}
\def\sgn{{\operatorname{sgn}}}
\def\Span{\operatorname{span}}
\def\Pr{\operatorname{Pr}}
\def\s{s}
\def\pot{\mathrm{pot}}
\def\curl{\mathrm{curl}}
\def\Vlg{\mathfrak{V}}
\def\s{\sigma}
\def\card{{\#}}
\def\opeps{-\diiv(a_\ve\nabla)}
\newcommand{\op}[1]{-\diiv(a\nabla #1)}

\def\uin{u^{\rm in}}
\def\dx{\,{\rm d}x}
\def\dy{\,{\rm d}y}
\def\pp{\partial}
\def\loc{{\rm loc}}
\def\ext{{\rm ext}}
\def\app{{\mathrm{app}}}


%\usetikzlibrary{positioning,arrows}
%\tikzset{main node/.style={circle,fill=blue!20,draw,minimum size=1cm,inner sep=0pt}}
 
\begin{document}
\title{Ramansh Sharma Written Qualifying Exam}
\date{}
\maketitle


\noindent Read the questions carefully. Follow the indicated page lengths. Feel free to cite references. You have one week. 

\begin{questions}
\question Let $0<m<d_1,d_2<\infty$ be integers.  
Let $(X_1,Y_1),\dots,(X_N,Y_N)$ be data points in $\R^{d_1}\times\R^{d_2}$.  

We want to derive a kernelized autoencoder.  
Given a kernel $K$ on $\R^{d_1}$ with RKHS $\H_K$ and a kernel $\Gamma$ on $\R^m$ with RKHS $\H_\Gamma$, we seek 
\[
  g=(g_1,\dots,g_m),\quad g_i\in \H_K, 
  \qquad 
  f=(f_1,\dots,f_{d_2}),\quad f_j\in\H_\Gamma,
\]
that minimize
\begin{equation}\label{eqautoencoder}
\min_{g_1,\ldots,g_m \in \H_K,\; f_1,\ldots,f_{d_2} \in \H_\Gamma} 
\sum_{i=1}^m \|g_i\|_K^2
+ \sum_{j=1}^{d_2}\|f_j\|_\Gamma^2
+ \lambda \,\big\| f\circ g(X)-Y\big\|_2^2,
\end{equation}
where $X=(X_1,\ldots,X_N)$, $Y=(Y_1,\ldots,Y_N)$, and $f\circ g(X)$ denotes the vector $(f(g(X_1)),\dots,f(g(X_N)))$.  
Reduce \eqref{eqautoencoder} to a finite-dimensional optimization problem.\\
{\bf Recommended answer length: less than a page. AI/LLM not allowed.}
%

{\bf Answer:}

In order to reduce~\eqref{eqautoencoder} to a finite-dimensional problem, we have to rewrite the functions $(g_1,\dots,g_m)$ and $(f_1,\dots,f_{d_2})$ in terms of their respective reproducing kernels. We assume kernels $K$ and $\Gamma$ are real valued kernels. By definition~\cite[Chapter 2.3]{fasshauer2015kernel}, the real valued functions from the RKHS $\H_K$ and $\H_\Gamma$ can respectively be written as,
\begin{align}
g_i(\cdot) &= \sum\limits_{p=1}^{N} c^i_p K(\cdot, \bx_p), \\
f_j(\cdot) &= \sum\limits_{k=1}^{m} d^j_k \Gamma(\cdot, \by_k),
\end{align}
where $i$ and $j$ denote the indices of the functions from the respective RKHS and $\bx \in \R^{d_1}, \by \in \R^m$ are arbitrary points in the respective domains of the kernels. Using the properties of symmetry and positive definiteness of reproducing kernels~\cite[Chapter 2.3]{fasshauer2015kernel}, and the forms of functions that belong to RKHS described above, we have the following Hilbert space norms,
\begin{align}
\|g_i\|_K^2 &= \left\langle g_i, g_i \right\rangle_K = \left\langle \sum\limits_{p=1}^{N} c^i_p K(\cdot, \bx_p), \sum\limits_{b=1}^{N} c^i_b K(\cdot, \bx_b) \right\rangle =  (\bc^i)^\Td \mathbf{K} \bc^i, \\
\|f_j\|_\Gamma^2 &= \left\langle f_j, f_j \right\rangle_{\Gamma} = \left\langle \sum\limits_{k=1}^{m} d^j_k \Gamma(\cdot, \by_k), \sum\limits_{t=1}^{m} d^j_t \Gamma(\cdot, \by_t) \right\rangle =  (\bd^j)^\Td \mathbf{\Gamma} \bd^j,
\end{align}
where, $\mathbf{K}$ and $\mathbf{\Gamma}$ are the full $(N\times N)$ and $(m\times m)$ kernel matrices respectively. The goal now is to rewrite~\eqref{eqautoencoder} as an optimization problem in terms of the expansion coefficients $\bc$ and $\bd$. We reformulate~\eqref{eqautoencoder} below,
\begin{align}
&\min_{\bc^1,\ldots,\bc^m,\; \bd^1,\ldots,\bd^{d_2}} 
\sum_{i=1}^m (\bc^i)^\Td \mathbf{K} \bc^i
+ \sum_{j=1}^{d_2} (\bd^j)^\Td \mathbf{\Gamma} \bd^j
+ \lambda \sum\limits_{n=1}^{N} \left( \begin{bmatrix}
f_1(g(X_n)) \\ 
f_2(g(X_n)) \\ 
\vdots \\
f_{d_2}(g(X_n))
\end{bmatrix} - Y_n \right)^2, \\
&\min_{\bc^1,\ldots,\bc^m,\; \bd^1,\ldots,\bd^{d_2}} 
\sum_{i=1}^m (\bc^i)^\Td \mathbf{K} \bc^i
+ \sum_{j=1}^{d_2} (\bd^j)^\Td \mathbf{\Gamma} \bd^j
+ \lambda \sum\limits_{n=1}^N \sum\limits_{j=1}^{d_2} \left( f_j(g(X_n)) - Y^j_n \right)^2, \label{eq:sumexpansion}
\end{align}
where $X_n$ and $Y_n$ are the $n^{\text{th}}$ data points respectively. For brevity, we use $\bg_n$ to denote $g(X_n) = \left(\mathbf{K}(X_n, X) \bc^1, \dots, \mathbf{K}(X_n, X) \bc^m \right)$, where $\mathbf{K}(X_n, X)$ is a column vector. Further, let $\mathfrak{g} = (\bg_1, \bg_2, \dots, \bg_N)$. Then,~\eqref{eq:sumexpansion} can be written as,
\begin{align}
\min_{\bc^1,\ldots,\bc^m,\; \bd^1,\ldots,\bd^{d_2}} 
\sum_{i=1}^m (\bc^i)^\Td \mathbf{K} \bc^i
+ \sum_{j=1}^{d_2} (\bd^j)^\Td \mathbf{\Gamma} \bd^j
+ \lambda \sum\limits_{n=1}^N \sum\limits_{j=1}^{d_2} \left( \mathbf{\Gamma}(\bg_n, \mathfrak{g}) \bd_j - Y^j_n \right)^2. \label{eq:finitedimopt}
\end{align}
In the final form shown in~\eqref{eq:finitedimopt}, the optimization problem is with respect to the finite-dimensional expansion coefficients $(\bc^1, \dots, \bc^m)$ and $(\bd^1, \dots, \bd^{d_2})$.


\question Consider the setting of operator learning, with an oracle/exact operator $G$:
\begin{align*}
  A \ni a \stackrel{G}{\mapsto} u \in U, \hskip 10pt
  G \in B(A; U), \hskip 5pt A = A(D_a; \R^a), \hskip 5pt U = U(D_u; \R^u),
\end{align*}
where $A$ and $U$ are function spaces (i.e., appropriate measure spaces, such as Hilbert or Banach spaces), with $a, u \in \N$. E.g., $A$ is a space of functions mapping some domain $D_a$ to $\R^a$, where $D_a$ is a subset of a Euclidean space. We likewise assume $B$ is some measurable space of mappings (operators) that map elements in $A$ to elements in $U$. We consider the \textit{supervised} learning setting, where $G$ is approximated through (possibly noisy) input-output pairs.

Informally, a \textit{neural operator} is a model class mapping class $A$ to $U$ that is constructed through iterative applications of affine global operators with componentwise/local nonlinear (``activation'') operators. Neural operator model classes with finite encodings are of particular interest, as they represent a feasible space of computable maps. The concept of \textit{universal approximation} is popular to investigate in neural operators, with the goal to establish that, e.g., given an arbitrary $G$ and tolerance $\epsilon$, there is a finite-encoding neural operator that is $\epsilon$-close to $G$.

Summarize the current state of universal approximation results/theorems in the literature, paying particular attention to pros and cons of these results (either individually or collectively). In particular, describe how (or if) one can use these universal approximation statements to guide computational construction of architectures. Propose a strategy to use some tools from existing methods for approximating functions over finite-dimensional spaces that might be used to provide more constructive and quantitatively rigorous methods for constructing neural operators.

In your response, consider the following aspects:
\begin{enumerate}
  \item Provide a reasonable high-level survey of current universal approximation results. Theoretical precision is appreciated, but the main goal is to summarize overall strengths and weaknesses of approximability statements. It's not important to cite every single result on universal approximation for neural operators, but provide enough breadth of narrative to cover ``much'' of the literature.
  \item Identify the practical utility of universal approximation results in the computational construction of neural operators. I.e., how would you use these results to actually construct computational architectures? (Are these results useful for that purpose?)
  \item Describe or propose how one might leverage existing quantitative results on approximation of functions to augment the current neural operator approximation theory. (Here, ``quantitative'' refers to actual, fairly precise rates of approximability or convergence for finite-dimensional approximation of functions.) In particular, how might one attempt to port quantitative results from function approximation to operators? What challenges require new investigations to address? It's perhaps most useful to narrow this discussion to leveraging a particular class/type of function approximation results, rather than attempting to broadly consider existing results on function approximation.
\end{enumerate}
{\bf Recommended answer length: 3-5 pages. AI/LLM not allowed.}
%

\pagebreak


\question {\bf I will ask two questions both related to your work. Answer one in detail (~1 page) and one in short (~1/2 page) (your choice).}

\begin{parts}
\part One of the challenges as I understand in SciML applications is that your error requirements are much more stringent than those in ML applications. For certain kinds of ML models, there are so-called "scaling laws" (e.g., \url{https://arxiv.org/pdf/2001.08361}). Go over the paper above, summarizing the main results. Would you expect analogous results for SciML applications (e.g., learning very simple PDEs' solutions), and if so, do you expect to drive the error to an arbitrarily small quantity? This is an open ended question, so please include references to existing work.

\part Study one of the early papers on "in context" learning, specifically: \url{https://arxiv.org/pdf/2208.01066}. While SciML papers do not think about what they do as ICL, expecting a model to learn a solution to a ``new'' PDE is quite similar in spirit. Go over the paper above and summarize the main results. There have been many follow up works to the paper above claiming that the distributions are important, etc., but one interesting paper is: \url{https://arxiv.org/pdf/2306.09927}.
\emph{Skim} the main results, but especially look at Section 4.2. Are you aware of analogous results in the SciML area? Does the distribution over boundary conditions, etc. matter for learnability?
\end{parts}
%
{\bf Answer:}

{\bf Part a:} \citep{kaplan2020scaling} is a study on neural scaling laws that investigates the effect of variables such as the model architecture, the model size (the network's width and depth for example), the computing power used to train, and the amount of training dataset on the overall performance. While related work exists that explore such trends in models such as random forests~\citep{biau2012analysis} and image models~\citep{tan2019efficientnet}, this study focuses on language models. The authors in~\citep{kaplan2020scaling} investigate and report power-law relationships between the performance (in terms of the loss on the test set) of the classical Transformer model~\citep{vaswani2017attention} on the WebText2 dataset~\citep{radford2019language} and three main variables of interest; the number of model parameters $N$, the dataset size $D$, and the amount of compute required $C$. Across the variety of the experiments conducted, the following trends are reported in the paper:
\begin{enumerate}
\item {\bf Power laws:} When not bottlenecked by the other two, the performance has a power-law relationship with each of the three factors $N, D, C$.
\item {\bf Overfitting:} If suboptimal values of $N$ or $D$ are used, the performance incurs a penalty as the scaling variable is increased.
\item {\bf Model size:} Large models tend to perform better overall; they need fewer optimization steps and a smaller dataset to achieve the same accuracy as smaller models. In fact, according to the authors the optimal performance is reached by training very large models and stopping significantly short of the convergence criteria.
\item {\bf Transfer learning:} Interestingly, the model's performance on a dataset from a different distribution than the training one is strongly correlated to the performance on the training set but with a constant offset.
\end{enumerate}
%
Since these scaling laws do not assume any special properties about the functions they are approximating, it is reasonable to expect these trends to carry over to SciML applications. While it is hard to find such studies done for physics-informed neural networks (architectures that can learn a given PDE solution function), a number of scaling studies are done in the field of operator learning, where many operators of interest are solution operators of PDEs. For example, the original DeepONet (deep operator network) paper~\citep{lu2019deeponet} showed various error convergence rates as a function of the amount of the network width, the amount of training data (number of functions), and the number of ``sensor locations'' (locations where the input functions are sampled).~\citep{de2022cost} reports power-law relationship between in- and out-of-distribution performance and the size of the training dataset for DeepONet, FNO, PCA-Net, and PARA-Net.~\citep{lanthaler2022error} studied the effect of the DeepONet network size on its approximation and generalization errors for specific operators. Finally,~\citep{liu2024neural} presents neural scaling laws (observed to be power laws) for DeepONets' approximation and generalization errors for general Lipschitz operators with respect to model and dataset sizes. While according to the scaling laws for operator learning methods the error can be driven down arbitrarily, the empirical results from~\citep{lu2019deeponet} show that this is not true. For example, Figure 6 in the paper shows that the relationship between training and test mean squared error with respect to the number of sensor locations initially follows a power law but plateaus after the number exceeds $10^1$. We see similar trends with the network width in Figure 2. This is consistent with the insights put forth by \citep{kaplan2020scaling}, that most scaling power laws plateau.
% \part One of the challenges as I understand in SciML applications is that your error requirements are much more stringent than those in ML applications. For certain kinds of ML models, there are so-called "scaling laws" (e.g., \url{https://arxiv.org/pdf/2001.08361}). Go over the paper above, summarizing the main results. Would you expect analogous results for SciML applications (e.g., learning very simple PDEs' solutions), and if so, do you expect to drive the error to an arbitrarily small quantity? This is an open ended question, so please include references to existing work.

{\bf Part b:} \citep{garg2022can} aims to study in-context learning of function classes with transformers. In-context learning is the ability of a model to, for a {\it previously unseen} function $f$, accurately predict $f(x_{i+1})$ at query point $x_{i+1}$ when given a prompt of $i$ ``in context'' sampling locations and samples of $f$, $\left(x_1, f(x_1), x_2, f(x_2), \dots, x_i, f(x_i) \right)$. For the class of linear functions, the transformer model can in-context learn well enough to be on par with several baselines; least squares estimator (known to be optimal), n-nearest neighbors, and averaging. In-context learning is also robust, it performs well even when the in-context outputs ($f(x_1), f(x_2), \dots, f(x_i)$) are noisy at inference time and when the distribution of the in-context examples and the query differ. The transformer model is also shown to in-context learn other function classes, mainly sparse linear functions, decision trees, and shallow neural networks. \citep{zhang2024trained} go a step further and show how in-context learning is ``learning a learning algorithm from data'' and study different distribution shifts under which the standard approach fails. Section 4.2 revisits three such shifts; task ($\D_{f}^{\text{train}} \neq \D_{f}^{\text{test}}$), query ($\D_{\text{query}}^{\text{test}} \neq \D_{x}^{\text{test}}$), and covariate ($\D_{x}^{\text{train}} \neq \D_{x}^{\text{test}}$). Transformers handle well the task and query shifts (the latter as long as $\D_{x}^{\text{train}} = \D_{x}^{\text{test}}$) but not covariate shifts. Interestingly, the same covariate shift phenomenon has been observed in SciML!

The original DeepONet paper~\citep{lu2019deeponet} states clearly that in order for the architecture to work, the sensor locations must be constant across all instances, essentially requiring $\D_{x}^{\text{train}} = \D_{x}^{\text{test}}$. This is generally true for most operator learning methods, though there is some work exploring ways to mitigate this by learning function encoders~\citep{ingebrand2025basis}. Most operator learning architectures also suffer from task shifts but none from query shifts (either at training or test time). When learning solution operators of PDEs, often the input functions are different initial and/or boundary conditions. As such, in order to avoid suffering from task shifts, the instances of input functions across training and test sets need to come from the same underlying distribution.
% \part Study one of the early papers on "in context" learning, specifically: \url{https://arxiv.org/pdf/2208.01066}. While SciML papers do not think about what they do as ICL, expecting a model to learn a solution to a ``new'' PDE is quite similar in spirit. Go over the paper above and summarize the main results. There have been many follow up works to the paper above claiming that the distributions are important, etc., but one interesting paper is: \url{https://arxiv.org/pdf/2306.09927}.
% \emph{Skim} the main results, but especially look at Section 4.2. Are you aware of analogous results in the SciML area? Does the distribution over boundary conditions, etc. matter for learnability?
\pagebreak


\question

\begin{parts}
\part Please introduce and summarize existing operator learning methods. Please give some categories, and summarize pros and cons for each category. \\
{\bf  Recommendation: no less than 2 pages.}
\part What do you think about the future of operator learning? You can explain whatever thoughts you have, positive, negative, and future development direction, etc. 
{\bf Recommendation: no less than 1 page.} 
\end{parts}
%

\pagebreak


\question Consider partition of unity (PoU) kernel methods in the context of divergence-free approximation. These methods allow one to scale global interpolation to very large numbers of points without a loss of computational efficiency. However, there are some specific challenges to applying PoU methods to divergence-free approximations. {\bf Note that we are not talking about operator learning, but just function approximation/interpolation}.\\
{\bf No AI/LLM allowed}.

\begin{parts}
\part It is quite straightforward to use a global divergence-free kernel for interpolation. However, in the PoU context, this is not straightforward at all. \emph{Show why} mathematically. \\
{\bf Recommendation: No more than 1/2 a page}.

\part Drake, Fuselier, and Wright present an approach in \url{https://arxiv.org/abs/2010.15898} that works for 2D approximations. Summarize their approach. What is the \emph{primary mathematical} difficulty in applying their technique to 3D problems? Hint:It relates to the nature of the div-free potentials in 3D, Eq 2.10.\\
{\bf Recommendation: No more than 1 page}.

\part How would you use machine learning (ML) to overcome these difficulties? Derive a new ML-based PoU technique to do so that is also divergence-free by construction; describe the ML architecture, point out how it overcomes the issue with the Drake-Fuselier-Wright method, discuss training procedures and difficulties. Briefly connect it to your answer in part 1. Assume you already have a div-free approximant/interpolant on each patch.\\
{\bf Recommendation: 1-2 pages}.
\end{parts}
%
{\bf Answer:}

{\bf Part a:} Let $\Omega \subset \R^{d}$ be a domain on which a vector-valued target function $\bof: \R^d \rightarrow \R^d$ is defined. Let $\phi: \R^d \times \R^d \rightarrow \R$ be a scalar valued kernel that is $C^2$-differentiable. We can then construct a \emph{matrix} valued divergence-free kernel $\Phidiv$ (whose columns are divergence free \emph{by construction}) as, $\Phidiv(\bx, \by) \coloneqq \curlx^{\top} \curly \; \phi(\bx, \by), \; \bx, \by \in \Omega$. Let $X = \{\bx_i \}\limits_{i=1}^{N}$ be a set of points in $\Omega$. Then, the global divergence-free interpolant $\bs$ is
\begin{align}
\bs(\bx) = \sum\limits_{i=1}^{N} \Phidiv(\bx, \bx_i) \; \bc_i, \label{divfreeinterp}
\end{align}
where $\bc_i \in \R^d$ are the interpolation coefficients. If instead we do partition-of-unity (PoU) interpolation,~\eqref{divfreeinterp} changes. We partition $\Omega$ into a set of $M$ overlapping patches $\{\Omega_j\}\limits_{j=1}^{M}$. The PoU approximant $\bspou$ is written as,
\begin{align}
\bspou(\bx) = \sum\limits_{k=1}^{M} \bs_k(\bx) \bc_k,
\end{align}
where $\bs_k(\bx) = w_k(\bx) \Phidiv(\bx, \bx_k)$, and $w_k$ are the compactly-supported blending functions. The problem here is that the basis functions $\bs_k$ are not divergence-free because of the multiplication with the weight functions. Hence, $\bspou$ is no longer divergence-free.

{\bf Part b:}~\citep{drake2021partition} used \emph{local} divergence-free and curl-free radial basis function (RBF) kernels to find the local scalar potential fields on each patch and blend them together with PoU to form a global scalar potential field. Then, the $\curlx^{\top} \curly$ operator is applied on the global scalar potential field to get the analytically divergence-free vector approximant. The key is to be able to extract a scalar potential $\psi$ out from~\eqref{divfreeinterp} in the following way (once the coefficients $\bc_i$ have been found),
\begin{align}
\bs(\bx) = \sum\limits_{i=1}^{N} \Phidiv(\bx, \bx_i) \bc_i = \sum\limits_{i=1}^{N} \left(\curlx^{\top} \curly \phi(\bx, \bx_i)\right) \bc_i = \underbrace{Q_{\bx} \nabla}_{\bL} \underbrace{\left(\sum\limits_{i=1}^{N} \nabla^{\top} \phi(\bx, \bx_i) Q_{\bx_i} \bc_i \right)}_{\psi(\bx)} = \bL(\psi(\bx)), \label{divfreeinterp_210}
\end{align}
where applying $Q_{\bx}$ to a vector in $\R^d$ gives the cross product of the unit normal $\mathbf{n}$ (in $d$ dimensions) with that vector, and $\psi$ is a scalar potential function that is unique up to an additive constant. $\bL = Q_{\bx} \nabla$ is the $\curlx$ operator. The idea behind the approach in this paper is to find and use the potential function $\psi_k$ from a patch's local divergence-free interpolant ${\bs_k^{_{\text{div}}}(\bx) = \sum\limits_{\forall \bx_t \in \Omega_k} \Phidiv(\bx, \bx_t) \bc_t}$. One approach is to say ${\tilde{\psi}(\bx) = \sum\limits_{k=1}^{M} w_k(\bx) \psi_k(\bx)}$ is the blended global PoU potential and then $\bL(\tilde{\psi})$ gives a global divergence-free approximant. The problem is that since the scalar potentials are unique only up to a constant, in the overlap regions the individual patch's scalar potential fields are not going to agree (they will be off up to the additive constant). Note that this problem does not occur in the normal PoU approximation because every patch's basis function is \emph{the same function} but only shifted (this is not true with the different $\psi_k$ calculated from~\eqref{divfreeinterp_210}). To rectify this, the authors shift each patch's $\psi_k$ by a constant $b_k$ such that $\psi_k + b_k \approx \psi_l + b_l$ for every patch $\Omega_l$ that overlaps with $\Omega_k$ (these constants $\{b_1, b_2, \dots, b_M\}$ need to be computed). Let $\tilde{\psi}_k$ denote a scaled and corrected \emph{local} potential function. Then, ${\tilde{\psi}(\bx) = \sum\limits_{k=1}^{M} w_k(\bx) \psi_k(\bx)}$ is the \emph{global} blended potential function. $\bL$ is then applied as follows to get a global PoU divergence-free approximant $\tilde{\mathbf{s}}_{_{\text{POU}}}$,
\begin{align}
\tilde{\mathbf{s}}^{_{\text{div}}}_{_{\text{POU}}}(\bx) \coloneqq \sum\limits_{k=1}^{M} \bL \left(w_k(\bx) \tilde{\psi}_k(\bx)\right) = \sum\limits_{k=1}^{M} w_k(\bx) \bs_k^{_{\text{div}}}(\bx) + \sum\limits_{k=1}^{M} \tilde{\psi}_k(\bx) \bL(w_k(\bx)). \label{divfreelocalpou}
\end{align}
%
The primary mathematical difficulty with extending this approach to 3D lies in~\eqref{divfreeinterp_210}. In 3D, the potential function associated with a divergence-free vector field is a vector field (this is due to the definition of the curl operator in 2D vs in 3D) that is unique up to an additive gradient of a harmonic scalar function. This disallows the key trick of this paper where only the difference between two potential functions' additive constants were accounted for in the overlapping regions.

{\bf Part c:} \citep{drake2021partition} solved the issue of $\bspou$ not being divergence-free. The main remaining difficulty is extending the approach to 3D problems. We derive a potential way to use machine learning to address this. The key idea is to approximate the vector-valued potential functions on each patch using a neural network. Let $\nn: \R^{3} \rightarrow \R^{3}$ denote a standard multilayer perceptron (MLP) neural network using some nonlinear activation function. We assume we already have the local divergence-free interpolants; $\bs_k^{_{\text{div}}}$ on the $k^{\text{th}}$ patch. Additionally, we use a different neural network on each patch denoted with $\nn_k$. We rewrite~\eqref{divfreelocalpou} as follows in this approach,
\begin{align}
\hat{\mathbf{s}}^{_{\text{div}}}_{_{\text{POU}}}(\bx) \coloneqq \sum\limits_{k=1}^{M} w_k(\bx) \bs_k^{_{\text{div}}}(\bx) + \sum\limits_{k=1}^{M} \nn_k(\bx) \bL(w_k(\bx)). \label{divfreelocalpou_nn}
\end{align}
Notice that we do not shift the networks $\nn_k$ yet for them to agree in the overlapping regions. It is not trivial nor cheap to find the gradient of the harmonic scalar function associated with $\nn_k$. Instead, we enforce that $\nn_k$ and $\nn_j$ for two patches $\Omega_k \cap \Omega_j \neq \emptyset$ (for an overlapping region of two patches) be equal to each other through a soft constraint. Let $\I_i$ denote the indices of the patches the point $\bx_i$ belongs to. Let $e_1$ be this loss function as follows,
\begin{align}
e_1 = \sum\limits_{i=1}^{N} \left\| \frac{1}{|\I_i|} \sum\limits_{k \in \I_i} \nn_k(\bx_i) \right\|_2.
\end{align}
For a given point, the expression inside the first sum goes to its minimum when all the neural networks have the same value, i.e. when all potential functions agree in the overlapping regions. Of course, this is not sufficient to train the networks since nothing is informing them about $\bof$. We use a second (supervised) loss function,
\begin{align}
e_2 = \sum\limits_{i=1}^{N} \left\| \hat{\mathbf{s}}^{_{\text{div}}}_{_{\text{POU}}}(\bx_i) - \bof(\bx_i) \right\|_2.
\end{align}
The total loss $\mathbf{e} = e_1 + e_2$ is then minimized using an off-the-shelf gradient based optimizer (eg: Adam, SGD, LBFGS). $\hat{\mathbf{s}}^{_{\text{div}}}_{_{\text{POU}}}$ can therefore overcome the limitation of the method in~\citep{drake2021partition} by way of machine learning. Additionally, since it is an extension of the~\citep{drake2021partition} method, it does not suffer from the problem in {\bf Part a}. $\hat{\mathbf{s}}^{_{\text{div}}}_{_{\text{POU}}}$ is a divergence-free approximant arising from the PoU blended vector potential function in $\R^3$.

{\bf Computational efficiency:} We can take certain implementation steps to compute the loss function efficiently at each training iteration. The weight functions $w_k$ are usually computed prior to training since they are not trainable. $\{\I_1, \I_2, \dots, \I_N\}$ can similarly be computed beforehand, with which we evaluate \emph{only} the $\nn_k$ and $\bs_{k}^{_{\text{div}}}$ for $k \in \I_i$ for every $\bx_i$. Finally, we can avoid computing $e_1$ on points that belong to a single patch.
%
\pagebreak


\end{questions}

\bibliographystyle{plainnat}
\bibliography{references}

\end{document}
